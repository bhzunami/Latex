%!TEX TS-program = xetex
%!TEX encoding = UTF-8 Unicode

% Um dieses Dokument zu erzeugen sind folgende Pakete erforderlich:
% - texlive-xetex (oft aber nicht immer Teil der TexLive-Distribution)
% - texlive-lang-german (auch nicht immer dabei)
% - Möglicherweise mehr TexLive-Pakete (siehe Logdatei nach dem Kompilieren).
% - Die Schrift "Linux Libertine".
% - Das Paket "libertine.sty". Achtung, nicht mit dem gleichnamigen LaTeX-Paket
%		verwechseln, welches unglücklicherweise standardmässig auch von XeTeX
%		gezogen wird. Es genügt das mitgelieferte Paket im gleichen Ordner
%		wie das Tex-Dokument bereitzustellen.
% - Die Schrift "Liberation Mono"
% - Das FHNW-Logo "fhnw-technik-head.eps"
%
% Das Dokument wird erzeugt mit: $ xelatex masterthesis-vorlage.tex
% Aufgrund der Querverweise sind nach Änderungen jeweils zwei xelatex-Läufe nötig.

\documentclass[paper=A4,twoside=false,BCOR=0mm,DIV=calc,fontsize=12pt]{scrartcl}

\usepackage[automark,headsepline]{scrpage2}
\usepackage{xunicode,fontspec,xltxtra}
\usepackage[english,german]{babel}
\usepackage{graphicx}
\usepackage{lastpage}
\usepackage[pagebackref=true,colorlinks=true,linkcolor=black,menucolor=black,urlcolor=black]{hyperref}
\usepackage{listings}

%\usepackage[debug]{libertine}
%\setromanfont[Mapping=tex-text]{Linux Libertine O}
%\setsansfont[Mapping=tex-text]{Linux Biolinum O}
%\setmonofont[Mapping=tex-text]{Liberation Mono}

\pagestyle{scrheadings}
\clearscrheadfoot
\ihead{\headmark}
\ohead{Seite\pagemark\ von \pageref{LastPage}}
\ifoot{Fusszeile}
\ofoot{\today}

\renewcommand*{\pnumfont}{
	\normalfont\rmfamily\slshape
}

\KOMAoptions{draft=true}
\KOMAoptions{DIV=last}

\begin{document}

% --- Titelseite --- %
\begin{titlepage}
	\enlargethispage{3cm}
	\begin{raggedright}
	\begin{picture}(0,0)
		\put(-30,14){\includegraphics[width=7cm]{fhnw-technik-head}}
	\end{picture}

	\vspace*{6cm}
	{\Huge\bfseries\sf
		Titel\\[1.7ex]
	}
	{\Large\bfseries\sf
		Untertitel\\[2.2ex]
	}
	{\large\bfseries\sf
		Masterthesis von\\[1.5ex]
		Vorname Name, Klasse\\[1.5ex]
	}
	\vspace*{1.5cm}
	{\large\bfseries\sf
		FHNW\\[1.5ex]
		Hochschule für Technik\\[1.5ex]
		Studiengang MAS-IT\\[1.5ex]
		Betreuender Dozent:\\[1.5ex]
		Dr. sc. techn. Ronald Tanner\\[1.5ex]
	}
	\vspace*{2cm}
	{\large\bfseries\sf
		Windisch, \today\\
	}
	\end{raggedright}
\end{titlepage}

\newpage
% --- Zusammenfassung --- %
\section*{Summary}
Text

% --- Vorwort --- %
\section*{Vorwort}
Text

% --- Inhaltsverzeichnis --- %
\newpage
	\tableofcontents

% --- Einleitung --- %
\newpage
\section{Einleitung}

\section{Überschrift 1}
\subsection{Überschrift 2}
\subsubsection{Überschrift 3}

\section{Schluss}

\end{document}

